% Chapitre sur le rapport de recherche :


\chapter{Research report} 

\section*{Introduction}

\subsection*{}
I am presenting in this chapter, the research project I have been working on during the PFE. I'have first undergone a bibliographic step, to acquire knowledge about the matter. Thatnks to this research, I could determine which problems are still to be solved. Then i tried to find appropriate solutions to the problems encountered.

The main difficulty in research, is that very often, there is no solution yet to the problem, in the domain. Therefore, I'have had to do a bibliography, in order to determine the state of the art, and eventually learn some new theories.

The image processing domain is particular: many algorithms are invented and published, but very few are available nor applicable to diverse images.
Thus, there is a re-programming and evaluation step, prior to any innovation.

\subsection*{}

While in the Megason Lab, I have been working on segmenting fluorescent microscopy images.
Datas are four-dimensional (space and time), and represent regions (ear, brain...) of a developing zebra fish.
It is possible to visualy distinc nuclei and membranes of cells. Those elements constitute the basis of the model that we try to create.
Therefore, we have to be able to detect and track every cell across time. The model will also have to integrate morphological informations of each cell.

The creation of this model is a thesis subject : cells lineage registration in microscopy, during which I would like to extend my work.

Prior to arriving in the laboratory, I have been working on cell membrane. I have then concentrated my researches on cell nuclei detection and localization.






\section{State of the art in the megason lab}


\subsection{Data analysis}

We are working on fluorescent microscopy images acquired with a 2-photons confocal microscope. The acquisition process produces huge datasets that we must process efficiently.
Here is a small description of the images :
\begin{figure}[htb]
\begin{center}
\begin{tabular}{|c|c|c|}
\hline Dimension & Size & Resolution \\ 
\hline x (space) & 1024 & 0.24 um \\ 
\hline y (space) & 1024 & 0.24 um \\ 
\hline z (space) & 70 & 1 um \\ 
\hline t (time) & 100 & 2 to 5 s \\ 
\hline \multicolumn{3}{|c|}{ 2 intensity channels} \\ 
\hline
\end{tabular} 
\end{center}
\caption{Megason Lab typical fluorescent microscopy dataset}
\label{tab:DataSizes}
\end{figure}
The table~\ref{tab:DataSizes}, shows that the datasets are huge ( the intensity values 





Kishore Mosaliganti has been working for two years in the Megason lab, in order to develop new segmentation methods for fluorescent images.

\subsection{membrane segmentation}









%
%
%
%
%
%\section{Segmentation de la membrane cellulaire par ensembles de niveaux}
%
%Le but initial du PFE etait la segmentation de la membrane cellulaire. il s'agit d'une fine membrane séparant les multiples cellules. Elle s'étend sur tout le spécimen à analyser. Il s'agit donc d'un volume important et complexe.
%
%\subsection{Étude du problème}
%
%J'ai tout d'abord cherché à comprendre le problème posé : sur quelles données allaient se baser la détection, existe-t'il des solutions pour segmenter ce genre de données.
%
%\subsubsection{Les données}
%Les images sont acquises a travers un système optique. L'excitation par un laser entraine la fluorescence de certaines parties de la cellule, marquées par une molécule émettant de la lumière dans un spectre dépendant du marqueur utilisé.
%
%Le système a donc une réponse impulsionelle bien visible dans les données. Un point correspond grossièrement a une gaussienne étalée dans les trois dimensions de l'espace, et plus particulièrement selon l'axe perpendiculaire au plan de focalisation.
%
%Il existe aussi un bruit dû au dispositif électronique d'acquisition. De plus, la fluorescence n'étant pas répartie de manière homogène, il existe des "trous" et de la saturation dans les données.
%\TODO{inserer des images illustrant les problemes}
%
%
%J'ai choisi de me focaliser sur trois difficultés afin de trouver des solutions :
%\begin{description}
%  \item [problème du bruit] : quel filtrage appliquer aux images, afin de les débruiter.
%  \item [problème de l'absence de données] : comment introduire des à priori de forme de la membrane
%  pour palier à l'absence d'information ?
%  \item [problème de la non homogénéité des intensités]  : comment segmenter un objet
%  qui n'occupe pas les mêmes intensités selon sa position dans l'espace.
%\end{description}
%
%
%\subsection{Débruitage des données}
%Le bruit présent sur les images n'est pas gaussien. Il n'est pas répartis de la même manière dans toute l'image non plus. Je me suis donc focalisé sur des techniques de débruitage telles que le filtre médian, et plus généralement des filtres morphologiques.
%
%Le filtre médian donne de bons résultats, pour un temps de calcul inférieur aux filtres morphologiques (reconstruction par dilatation/erosion)
%
%\subsection{Segmentation de la membrane}
%
%\subsubsection{Utilisation de la théorie des ensembles de niveaux}
%
%L'outil choisi pour segmenter la paroi cellulaire, est base sur les ensembles de niveaux (levels-sets). Cette théorie consiste en l'évolution d'un front. Cette évolution est représentée par une fonction implicite qui évolue itérativement. Le front (bords de la zone segmentée) est souvent représenté par le niveau zéro de cette fonction implicite.
%Les Level sets, au travers de leur critère d'évolution, permettent d'avoir une grande flexibilité quand aux mesures a considerer lors de l'evolution du front. Cette évolution est représentée par un critère d'énergie, le problème de segmentation par level set est donc un problème d'optimisation.
%
%\subsection{des idees}
%
%
%
%
%
%
%idee de la mediane
%idee morphologie
%idee localisation
%\subsection{resultats}
%idee mediane
%idee morphologie
%idee localisation
%\subsection{travail futur}
%rapidite
%
%
%
%\section{Detection et localisation des cellules}
%
%Nous basons nos méthodes de segmentation sur une initialisation au centre des cellules. Nous avons donc besoin de détecter un maximum de cellules afin de trouver un point a l'intérieur de ces dernières. Des methodes ont ete proposees, cependant, chacune est adaptee a un type d'image particulier.
%Cels algorithmes de détection sont aussi souvent appelles algorithmes de "seeding" car ils permettent d'obtenir des points a partir desquels une segmentation peut etre initialisee, afin de delimiter les bordures des noyaux, ou les membranes cellulaires.
%
%\subsection{demarche}
%
%Nous avons developpe une methode combinant l'information provenant des noyaux et de la membrane des cellules. Cette methode doit etre evaluee, donc comparee a d'autres methodes existantes. Ce processus d'evaluation nous permettra aussi de trouver les points forts et les points faibles des algorithmes. Nous pourrons ainsi eventuellement utiliser des techniques de fusion d'information pour combiner les resultats de differents algorithmes.
%La creation d'un "framework" d'evaluation passe donc par plusieurs etapes : l'implementation des algorithmes existants, afin de les tester sur des images synthetiques puis reelles, la creation de criteres d'evaluation appropories, et l'observation des resultats. Nous avons aussi initié un travail afin de proposer une nouvelle methode de detection de cellules basee sur la decomposition en ondelettes.
%
%
%\subsection {description des algorithmes evalues}
%
%
%
%\subsubsection{chaine de traitement de l'image}
%
%partie commune
%Nous nous focalisons sur une classe d'algorithmes traitant l'information issue de l'image des noyaux cellulaires, apres une detection des zones d'interet (binarisation de l'image). Ces algorithmes fonctionnent aussi souvent avec une extraction de maxima locaux en dernier traitement.
%Nous choisissons d'utiliser la même binarisation, et la meme methode d'extraction de maximas locaux pour les deux algorithmes afin de focaliser l'etude sur la technique de detection des centres des noyaux.
%
%\subsubsection{description des algorithmes}
%\paragraph{le Laplacien de la Gaussienne ameliore}
%Nous avons decide d'implementer l'algorithme presente dans \cite{al2009improved}. La methode utilisee est celle du Laplacien de la Gaussienne (LoG). Une methode eprouveee qui s'est montree tres robuste dans d'autres applications telles la detection de points de reperes pour le recallage photographique.
%
%
%
%\paragraph{Kishore}
%
%\TODO{Ask Kishore more infos}
%
%
%\subsection {evaluation}
%%\begin{tabular}{|c|c|c|c|c|}
%%\hline  & Matching & UnMatching & Missed & Accuracy \\ 
%%\hline A1 & 10 & 3 & 1 & 71% \\ 
%%\hline A2 & 9 & 2 & 3 & 64% \\ 
%%\hline 
%%\end{tabular} 
%
%
%
%\subsection {conclusion}
%
%
%\subsection {proposition}
%
%
%
%\subsection {planning}
%
%
%\subsection{resultats}
%
%\subsection{proposition}
%
%
%
%
